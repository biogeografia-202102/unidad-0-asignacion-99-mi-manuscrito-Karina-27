\documentclass[11pt,]{article}
\usepackage[left=1in,top=1in,right=1in,bottom=1in]{geometry}
\newcommand*{\authorfont}{\fontfamily{phv}\selectfont}
\usepackage[]{mathpazo}


  \usepackage[T1]{fontenc}
  \usepackage[utf8]{inputenc}



\usepackage{abstract}
\renewcommand{\abstractname}{}    % clear the title
\renewcommand{\absnamepos}{empty} % originally center

\renewenvironment{abstract}
 {{%
    \setlength{\leftmargin}{0mm}
    \setlength{\rightmargin}{\leftmargin}%
  }%
  \relax}
 {\endlist}

\makeatletter
\def\@maketitle{%
  \newpage
%  \null
%  \vskip 2em%
%  \begin{center}%
  \let \footnote \thanks
    {\fontsize{18}{20}\selectfont\raggedright  \setlength{\parindent}{0pt} \@title \par}%
}
%\fi
\makeatother




\setcounter{secnumdepth}{3}



\title{Título\\[2\baselineskip]  }



\author{\Large Melany Karina Ogando Matos\vspace{0.05in} \newline\normalsize\emph{Estudiante, Universidad Autónoma de Santo Domingo (UASD)}  }


\date{}

\usepackage{titlesec}

\titleformat*{\section}{\normalsize\bfseries}
\titleformat*{\subsection}{\normalsize\itshape}
\titleformat*{\subsubsection}{\normalsize\itshape}
\titleformat*{\paragraph}{\normalsize\itshape}
\titleformat*{\subparagraph}{\normalsize\itshape}

\titlespacing{\section}
{0pt}{36pt}{0pt}
\titlespacing{\subsection}
{0pt}{36pt}{0pt}
\titlespacing{\subsubsection}
{0pt}{36pt}{0pt}





\newtheorem{hypothesis}{Hypothesis}
\usepackage{setspace}

\makeatletter
\@ifpackageloaded{hyperref}{}{%
\ifxetex
  \PassOptionsToPackage{hyphens}{url}\usepackage[setpagesize=false, % page size defined by xetex
              unicode=false, % unicode breaks when used with xetex
              xetex]{hyperref}
\else
  \PassOptionsToPackage{hyphens}{url}\usepackage[unicode=true]{hyperref}
\fi
}

\@ifpackageloaded{color}{
    \PassOptionsToPackage{usenames,dvipsnames}{color}
}{%
    \usepackage[usenames,dvipsnames]{color}
}
\makeatother
\hypersetup{breaklinks=true,
            bookmarks=true,
            pdfauthor={Melany Karina Ogando Matos (Estudiante, Universidad Autónoma de Santo Domingo (UASD))},
             pdfkeywords = {palabra clave 1, palabra clave 2},  
            pdftitle={Título\\[2\baselineskip]},
            colorlinks=true,
            citecolor=blue,
            urlcolor=blue,
            linkcolor=magenta,
            pdfborder={0 0 0}}
\urlstyle{same}  % don't use monospace font for urls

% set default figure placement to htbp
\makeatletter
\def\fps@figure{htbp}
\makeatother

\usepackage{pdflscape} \newcommand{\blandscape}{\begin{landscape}}
\newcommand{\elandscape}{\end{landscape}} \usepackage{float}
\floatplacement{figure}{H}
\newcommand{\beginsupplement}{ \setcounter{table}{0} \renewcommand{\thetable}{S\arabic{table}} \setcounter{figure}{0} \renewcommand{\thefigure}{S\arabic{figure}} }


% add tightlist ----------
\providecommand{\tightlist}{%
\setlength{\itemsep}{0pt}\setlength{\parskip}{0pt}}

\begin{document}
	
% \pagenumbering{arabic}% resets `page` counter to 1 
%
% \maketitle

{% \usefont{T1}{pnc}{m}{n}
\setlength{\parindent}{0pt}
\thispagestyle{plain}
{\fontsize{18}{20}\selectfont\raggedright 
\maketitle  % title \par  

}

{
   \vskip 13.5pt\relax \normalsize\fontsize{11}{12} 
\textbf{\authorfont Melany Karina Ogando Matos} \hskip 15pt \emph{\small Estudiante, Universidad Autónoma de Santo Domingo (UASD)}   

}

}








\begin{abstract}

    \hbox{\vrule height .2pt width 39.14pc}

    \vskip 8.5pt % \small 

\noindent Resumen del manuscrito


\vskip 8.5pt \noindent \emph{Keywords}: palabra clave 1, palabra clave 2 \par

    \hbox{\vrule height .2pt width 39.14pc}



\end{abstract}


\vskip 6.5pt


\noindent  \section{Introducción}\label{introducciuxf3n}

Conocer la dispersion y las formas de agrupamiento de los individuos de
una especie, es necesario para su conservacion (Condit et al., 1996,
Hubbell \& Foster (1992)). Las especies no son las mismas en todos los
lugares, y estos patrones de variación, en cada uno de los niveles de
diversidad: genes, especies y ecosistemas, es el objeto de estudio de la
biogeografía. Su objetivo es caracterizar la distribucion de las
especies en la actualidad y la variacion geografica de la diversidad en
términos de la interacción de los organismos con su ambiente (Lomolino,
Riddle, Whittaker, \& Brown, 2010). La similiridad es una medida simple
de la similitud de especies y sus abundancias. Es convencional decir que
es lo mismo alta diversidad con alta homogeneidad, lo que es equivalente
a poca dominancia (Magurran, 2004).

Los patrones de biodiversidad son el resultado de la combinacion de los
procesos internos de la comunidad de plantas y las condiciones externas
del ambiente (Sang \& Bai, 2008). Por ejemplo, la cantidad de nitrogeno
en el suelo puede ser asumida como una limitante directa en la
distribucion de las especies de plantas (Lange, Nobel, Osmond, \&
Ziegler, 2013). Numerosas especies de la familia Chrysobalanaceae poseen
preferencia por suelos húmedos (Regional Conservation (2020), Future
(2021), C. Sothers, Prance, Buerki, De Kok, \& Chase (2014), Grandtner
\& Chevrette (2013)).Esta es una familia de plantas de distribucion
pantropical, y cuenta con 18 géneros con 531 especies que se encuentra
un 80\% en el neotrópico.(G. Prance, 2014, Bardon et al. (2013)).

Las especies de Chrysobalaneceae son utilizadas de maneras distintas
para el tratamiento y como medicina de algunas enfermedades como la
malaria, epilepsia, diarrea y diabetes (Feitosa, Xavier, \& Randau,
2012). Sus usos son frecuentes en la región africana y surámericana,
donde son más abundantes (Feitosa et al., 2012). Posee diferentes usos
como: El aceite de sus frutos para pinturas y varnices, tambien su
madera como material de construccion, combustible y carbón. Ademas es
utilizada mezclada con arcilla para hacer vasijas de barro (G. Prance,
2014).

La parcela permanente de la isla de Barro Colorado es una reserva de
investigacion biologica a cargo del Smithsonian Tropical Research
Institute (Croat, 1978).\\
Segun Condit (1998) la distribucion de \emph{Hirtella americana} en la
parcela permanente de BCI se encuentra de manera irregular en pequeñas
areas isoladas. Sin embargo, los estudios realizados no han demostrado
que esta irregularidad se deba a variables ambientales. Croat (1978)
menciona que \emph{Hirtella americana} es una de las especies mayor
representadas por su densidad en este bosque.

Con el motivo de aplicar estudios de ecología numérica y aprender a
utilizarlos en futuras investigaciones, además de descubrir si existen
patrones de correlación entre la familia de plantas Chrysobalanaceae y
algunas variables ambientales, esta investigación tiene como objetivos
conocer la distribución espacial de las especies de la familia
Chrysobalanaceae, conoces si se agrupan en grupos discontinuos en
función de su composición, y reconocer si estos patrones se encuentran
relacionados con alguna variable ambiental. Además de identificar, en
caso de que existan, las tendencias de ordenación de la especies de
Chrysobalanaceae representadas en BCI y su correspondencia con las
variables del ambiente. Este estudio busca estimar la riqueza de la
familia Chrysobalanceae en la parcela permanente de Barro Colorado
Island y descubir la relación entre la diversidad alpha y las variables
ambientales, además de la relación de las especies con la diversidad
beta. Tambien identificar si los modelos de distribución de especies
predicen la ocurrencia de las especies representadas en el estudio.

Para realizar un estudio a fondo de la correlación entre esta familia y
las variables ambientales era necesario realizar multiples analisis de
ecología numerica, por esta razon se realizaron analisis de
agrupamiento, tecnicas de ordenacion, analaisis de diversidad y ecología
espacial. Ya que cada uno responde a distintas preguntas de
investigación. Este estudio intenta resumir los patrones de diversidad
asociados a la familia Chrysobalanaceae en la parcela permanente de BCI.

\section{Metodología}\label{metodologuxeda}

\section{Resultados}\label{resultados}

\section{Discusión}\label{discusiuxf3n}

\section{Agradecimientos}\label{agradecimientos}

\section{Información de soporte}\label{informaciuxf3n-de-soporte}

\section{\texorpdfstring{\emph{Script}
reproducible}{Script reproducible}}\label{script-reproducible}

\section*{Referencias}\label{referencias}
\addcontentsline{toc}{section}{Referencias}

\hypertarget{refs}{}
\hypertarget{ref-bardon2013origin}{}
Bardon, L., Chamagne, J., Dexter, K. G., Sothers, C. A., Prance, G. T.,
\& Chave, J. (2013). Origin and evolution of chrysobalanaceae: Insights
into the evolution of plants in the neotropics. \emph{Botanical Journal
of the Linnean Society}, \emph{171}(1), 19--37.

\hypertarget{ref-condit1998tropical}{}
Condit, R. (1998). \emph{Tropical forest census plots: Methods and
results from barro colorado island, panama and a comparison with other
plots}. Springer Science \& Business Media.

\hypertarget{ref-condit1996species}{}
Condit, R., Hubbell, S. P., Lafrankie, J. V., Sukumar, R., Manokaran,
N., Foster, R. B., \& Ashton, P. S. (1996). Species-area and
species-individual relationships for tropical trees: A comparison of
three 50-ha plots. \emph{Journal of Ecology}, 549--562.

\hypertarget{ref-croat1978flora}{}
Croat, T. B. (1978). \emph{Flora of barro colorado island}. Stanford
University Press.

\hypertarget{ref-feitosa2012chrysobalanaceae}{}
Feitosa, E. A., Xavier, H. S., \& Randau, K. P. (2012).
Chrysobalanaceae: Traditional uses, phytochemistry and pharmacology.
\emph{Revista Brasileira de Farmacognosia}, \emph{22}(5), 1181--1186.

\hypertarget{ref-PfafLrigida}{}
Future, P. for a. (2021). Licania rigida (benth). Retrieved November 11,
2021, from
\url{https://pfaf.org/user/Plant.aspx?LatinName=Licania+rigida}

\hypertarget{ref-grandtner2013dictionary}{}
Grandtner, M. M., \& Chevrette, J. (2013). \emph{Dictionary of trees,
volume 2: South america: Nomenclature, taxonomy and ecology}. Academic
Press.

\hypertarget{ref-hubbell1992short}{}
Hubbell, S. P., \& Foster, R. B. (1992). Short-term dynamics of a
neotropical forest: Why ecological research matters to tropical
conservation and management. \emph{Oikos}, 48--61.

\hypertarget{ref-lange2013physiological}{}
Lange, O. L., Nobel, P. S., Osmond, C. B., \& Ziegler, H. (2013).
\emph{Physiological plant ecology iii: Responses to the chemical and
biological environment} (Vol. 12). Springer Science \& Business Media.

\hypertarget{ref-lomolino2017biogeography}{}
Lomolino, M. V., Riddle, B. R., Whittaker, R. J., \& Brown, J. H.
(2010). \emph{Biogeography}.

\hypertarget{ref-magurran2004measuring}{}
Magurran, A. (2004). Measuring biological diversity. 2004. \emph{Malden:
Blackwell}.

\hypertarget{ref-prance2014chrysobalanaceae}{}
Prance, G. (2014). Chrysobalanaceae. In \emph{Flowering plants.
eudicots} (pp. 19--28). Springer.

\hypertarget{ref-ircCicaco}{}
Regional Conservation, T. I. for. (2020). Chrysobalanus icaco,
chrysobalanaceae. Retrieved November 11, 2021, from
\url{https://regionalconservation.org/beta/nfyn/plantdetail.asp?tx=Chryicac\&tx=Chryicac}

\hypertarget{ref-sang2008vascular}{}
Sang, W., \& Bai, F. (2008). Vascular diversity patterns of forest
ecosystem before and after a 43-year interval under changing climate
conditions in the changbaishan nature reserve, northeastern china. In
\emph{Forest ecology} (pp. 115--130). Springer.

\hypertarget{ref-sothers2014taxonomic}{}
Sothers, C., Prance, G. T., Buerki, S., De Kok, R., \& Chase, M. W.
(2014). Taxonomic novelties in neotropical chrysobalanaceae: Towards a
monophyletic couepia. \emph{Phytotaxa}, \emph{172}(2), 176--200.




\newpage
\singlespacing 
\end{document}
